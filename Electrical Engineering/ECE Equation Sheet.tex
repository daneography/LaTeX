\documentclass [12pt] {article}
\usepackage{graphicx}
\usepackage{amsmath}
\usepackage{multicol}
\usepackage{gensymb}
\pagestyle{empty}
\setlength{\topmargin}{-1.2in}
\setlength{\textheight}{9.5in}
\setlength{\oddsidemargin}{-.6in}
\setlength{\textwidth}{7in}
\begin{document}
ECE 2070 Equation Sheet \hfill Test 1 \\
\rule{7.6in}{0.4pt}
\begin{multicols*}{2}
\section{Kirchoff's Current Law} 
	\begin{itemize}
		\item $\sum\limits_{n=1}^N i = 0$ 
		\item Current pointing into a node is negative, pointing out is positive
	\end{itemize}
\section{Kirchoff's Voltage Law}
	\begin{itemize}
		\item $\sum\limits^N \pm v_n = 0$
		\item $\Delta V = IR$
		\item Voltage rise (+V) means entering from - and going to +
		\item Voltage drop (-V) enters at + and leaves at -
	\end{itemize}
\section{Resistors}
\textbf{Series}
	\begin{itemize}
		\item $R_{series} = R_1 + R_2 + R_3 + \cdots$
		\item $Q = Q_1 = Q_2 = Q_3 = \cdots $
		\item $\Delta V = \Delta V_1 + \Delta V_2 + \Delta V_3 + \cdots$
		\item Current same everywhere, voltage breaks up
	\end{itemize}
\textbf{Parallel} 
	\begin{itemize}
		\item $R_{parallel} = \left(\frac{1}{R_1} + \frac{1}{R_2} + \frac{1}{R_3} + \cdots\right)^{-1} $
		\item $Q = Q_1 + Q_2 + Q_3 + \cdots $
		\item $\Delta V = \Delta V_1 = \Delta V_2 = \Delta V_3 = \cdots $
		\item Voltage same everywhere, current breaks up 
	\end{itemize}
\section{Power}
	\begin{itemize}
		\item $P = I\Delta V = RI^2 = \frac{\Delta V^2}{R}$ 
		\item When current flows into +, power is dissipated (P = IV) 
		\item When current flows into -, power is supplied (P = -IV) 
		\item 1 W = 1 V$\cdot$A 
	\end{itemize}
\section{Energy}
	\begin{itemize}
		\item $E = vi\Delta t ($Watt$ \cdot $seconds)
		\item $V\cdot Ah$ approx. = Ah when voltage is implied
	\end{itemize}
\section{Nodal Analysis}
	\begin{itemize}
		\item $i_{node} = \frac{v_{node} - v_{adjacent}}{R}$ 
		\item $\sum v_n = (i_{mesh} - i_{adjacent})R $ (mesh current) 
		\item Node voltages uses KCL, mesh current uses KVL 
	\end{itemize}
\section{Equivalent Circuits}
	\begin{itemize}
		\item Thevenin is a single voltage source and a resistor in series 
		\item Norton is a single current source and a resistor in parallel 
		\item Max resistance= $P_{max} = \frac{V_{th}^2}{4R_{th}}$ 
	\end{itemize}
\section{Misc.}
	\begin{itemize}
		\item Number of equations = $n - m - 1$ \\
		\item $I = \frac{dq}{dt}$ \\
		\item $I_{avg} = \frac{\Delta Q}{\Delta t}$ \\
		\item $R = \frac{\rho l}{A} = \frac{L}{\sigma A}$ \\
		\item $\Delta V_{bat} = \epsilon - Ir$ \\
		\item $\sum_{loop}\Delta V = 0$ \\
		\item $\sum I_{in} = \sum I_{out}$ \\ 
		\item $V_1 = \frac{R_n}{R_1 + R_2 + \dots}V_{tot} $ (voltage divider) \\
		\item $I_n = \frac{\left(\frac{1}{R_1} + \frac{1}{R_2} + \dots\right)^{-1}}{R_n}I_{tot}$ (current divider) \\
		\item If current comes from + terminal, $V = -iR$, from -, $V = iR$ \\
		\item No current flows through open circuit or switch \\
		\item Short circuit- $R = 0$ \\
		\item \underline{\textbf{Constant voltage source}}- produces a constant voltage not affected by other components, current determined by connections to other components and can supply any current (car battery)\\
		\item \underline{\textbf{Constant current source}}- produces a constant current not affected by connections to other components, voltage determined by connections to other components and can supply any voltage (cell phone charger) \\
		\item \underline{\textbf{Terminal}}- point where a component or part of the circuit connects to other components of the circuit \\
		\item \underline{\textbf{Node}}- connection point \\
		\item \underline{\textbf{Branch}}- portion of a circuit with only two external terminals \\
		\item \underline{\textbf{Loop}}- a closed connection of branches \\
		\item \underline{\textbf{Mesh}}- a loop that does not contain other loops \\
		\item \underline{\textbf{Voltage}}- work done to move charge between two points \\
		\item Mesh is a loop but a loop is not necessarily a mesh
	\end{itemize}
\end{multicols*}
\newpage
ECE 2070 Equation Sheet \hfill Test 2 \\
\rule{7.6in}{0.4pt}
\begin{multicols*}{2}
\section{Equivalent Circuits}
	\begin{itemize}
		\item Any part of the circuit with two terminals can be replaced by a single voltage source and a resistor in series for Thevenin 
		\item Find $R_{eq}$ and find open circuit voltage for Thevenin
		\item Any part of the circuit with two terminals can be replaced by a single current source and a resistor in parallel for Norton 
	\end{itemize}
\section{Loops}
	\begin{itemize}
		\item For analysis, $n - m - 1$ equations where $m$ is the number of dependent loops
		\item $i_{node} = \frac{V_{node} - V_{adjacent}}{R}$
		\item For mesh current, $\sum V_n$ around mesh
		\item $(i_{mesh} - i_{adjacent})R$
	\end{itemize}
\section{Short and Open Circuits}
	\begin{itemize}
		\item Voltage source is a short circuit
		\item Current source is an open circuit
	\end{itemize}
\section{Misc.}
	\begin{itemize}
		\item $P_{max} = \frac{V_{th}^2}{4R_{th}}$
	\end{itemize}
\end{multicols*}
\newpage
\section{Blank Space}
\newpage
ECE 2070 Equation Sheet \hfill Test 3 \\
\rule{7.6in}{0.4pt}
\begin{multicols*}{2}
\section{Capacitors}
	\begin{itemize}
		\item $C = \frac{\epsilon_0\epsilon_rA}{D}$ Farad (F) (D = plate distance, A = plate area, $\epsilon_0\epsilon_r$ = property of material between plates)
		\item Capacitors combined opposite of resistors
	\end{itemize}
\textbf{Time Domain} 
	\begin{itemize}
		\item $i_c(t) = C\frac{dv_c(t)}{dt}$  
		\item $p(t) = v(t)C\frac{dv(t)}{dt} $
		\item $W = \frac{1}{2}Cv^2(t)$
		\item $v_s(t) = V_{max}\cos(\omega t) $
		\item $i = CV_{max}\omega\cos(\omega t + \frac{\pi}{2}) $
	\end{itemize}
\textbf{Frequency Domain} 
	\begin{itemize}
		\item $V_C(j\omega)  = V_S(j\omega) = V_{max}\angle 0\degree$ 
		\item $I_C(j\omega) = CV_{max}\omega\angle 90\degree$
		\item $\frac{V_C(j\omega)}{I_C(j\omega)} = \frac{1}{C\omega}\angle -90\degree \Omega $
	\end{itemize}
\section{Inductors}
	\begin{itemize}
		\item $L = \frac{\mu_0\mu_rN^2A}{l}$ Henry (H) (A = coil area, l = coil height, $\mu_0\mu_r$ = property of core material, N = number of coil turns) 
		\item Inductors combined same as resistors 
	\end{itemize}
\textbf{Time Domain} 
	\begin{itemize}
		\item $v_L(t) = L\frac{di_L(t)}{dt}$
		\item $p(t) = L\frac{di(t)}{dt}i(t)$
		\item $ W = \frac{1}{2}Li^2(t) $ 
		\item $v_s(t) = V_{max}\cos(\omega t) $
		\item $i = \frac{V_{max}}{\omega L}\cos(\omega t - \frac{\pi}{2}) $
	\end{itemize}
\textbf{Frequency Domain}
	\begin{itemize}
		\item $V_L(j\omega)  = V_S(j\omega) = V_{max}\angle 0\degree$
		\item $I_L(j\omega) = \frac{V_{max}}{\omega L}\angle -90 \degree$
		\item $\frac{V_L(j\omega}{I_L(j\omega)} = \omega L \angle 90 \degree \Omega $
	\end{itemize}
\section{Resistors}
\textbf{Time Domain} 
	\begin{itemize}
		\item $v_s(t) = V_{max}\cos(\omega t)$
		\item $i = \frac{V_{max}}{R}\cos(\omega t)$
	\end{itemize}
\textbf{Frequency Domain} 
	\begin{itemize}
		\item $V_R(j\omega) = V_S(j\omega) = V_{max}\angle0\degree $
		\item $I_R(j\omega) = \frac{V_{max}}{R}\angle 0\degree$ 
		\item $\frac{V_R(j\omega)}{I_R(j\omega)} = \frac{V_{max}\angle 0\degree}{\frac{V_{max}}{R}\angle 0\degree} = R\angle 0 \degree \Omega $
	\end{itemize}
\section{Power \& Energy}
	\begin{itemize}
		\item Power can be absorbed or supplied for both
		\item All energy dissipated as heat for resistor, stored in magnetic field for inductor (0 if i = 0), electric field for capacitor (0 if v = 0)
	\end{itemize}
\section{AC Sources}
	\begin{itemize}
		\item $f = \frac{1}{T}$ Hz (frequency) 
		\item $\omega = 2\pi f = \frac{2\pi}{T} $ rad/s 
		\item $\phi = 2\pi\frac{\Delta t}{T}$ rad $ = 360\frac{\Delta t}{T}$ deg (phase shift)
		\item $x(t) = A\cos(\omega t + \phi)$ (A = amplitude)
		\item $V_{rms} = \sqrt{\frac{1}{T}\int_0^Tx^2(t)dt} = \frac{V_{max}}{\sqrt{2}} $
	\end{itemize}
\section{Phasors}
	\begin{itemize}
		\item $v(t) = A\cos(\omega_0t + \theta) \rightarrow V(j\omega) = Ae^{j\theta} = A\cos\theta + jA\sin\theta = A \angle\theta $
	\end{itemize}
\section{Impedance}
	\begin{itemize}
		\item $Z(j\omega) = \frac{V(j\omega)}{I(j\omega)} $
		\item $V_T = I_NZ$ 
		\item Combined similar to resistors for total Z 
		\item Thevenin equivalent combines a single voltage source and impedance in series 
		\item Norton equivalent consists of a single current source and impedance in parallel 
		\item For resistors, $Z_R = R = R \angle 0\degree = R \angle 0$
		\item For inductors, $Z_L = j\omega L = \omega L\angle 90\degree = \omega L \angle \frac{\pi}{2}$
		\item For capacitors, $Z_C = \frac{1}{j\omega C} = \frac{1}{C\omega}\angle -90\degree = \frac{1}{C\omega}\angle -\frac{\pi}{2}$
	\end{itemize}
\section{Misc.}
	\begin{itemize}
		\item Complex rect. form - $x = A + jB$ 
		\item $C = \sqrt{A^2 + B^2}$ 
		\item $\theta = tan^{-1}(\frac{B}{A})$ 
		\item $x = Ce^{j\theta} = C \angle \theta $ (polar form) 
		\item $-180\degree < \theta \leq 180\degree $ or $ -\pi < \theta \leq \pi$ 
		\item $x = C\cos\theta + jC\sin\theta$ (rectangular form) 
		\item $V_1 \cdot V_2 = C_1\angle\theta_1 \cdot C_2\angle\theta_2 = C_1C_2\angle(\theta_1+\theta_2)$ 
		\item $\frac{V_1}{V_2} = \frac{C_1\angle\theta_1}{C_2\angle\theta_2} = \frac{C_1}{C_2}\angle(\theta_1 - \theta_2)$ 
	\end{itemize}
\end{multicols*}
\newpage
ECE 2070 Equation Sheet \hfill Test 4 \\
\rule{7.6in}{0.4pt}
\begin{multicols*}{2}
\section{Power}
	\begin{itemize}
		\item Instantaneous power - $p(t) = \frac{v^2(t)}{R}$
		\item Average power - $\frac{1}{T} \int_0^T p(t) dt = \frac{1}{2}VI\cos(\theta_V - \theta_I) = V_{rms}I_{rms}\cos(\theta_V - \theta_I)$
		\item Placeholder %Include a picture of reactive power triangle here
		\item Above image uses RMS values
		\item Complex power - $P + Qj$
		\item P is in watts, Q is in VAR, S is in VA
		\item $S = \tilde{V}I^*$
		\item S and Z (impedance) have the same angle
		\item Impedance angle is the phase shift between voltage and current
	\end{itemize}
\section{Voltage}
	\begin{itemize}
		\item Mean square voltage - $v^2 = \frac{V^2}{2}$
		\item RMS voltage - $V_{rms} = \frac{V}{\sqrt{2}}$
	\end{itemize}
\section{Transformers}
	\begin{itemize}
		\item $N = \frac{n_2}{n_1}$
		\item $\frac{V_2}{n_2} = \frac{V_1}{n_1}$
		\item $n_2I_2 = n_1I_1$
		\item $V_2 = NV_1$
		\item $I_2 = \frac{I_1}{N}$
		\item $Z_{ab} = \frac{1}{N^2}Z_L$
		\item $R_L = N^2R_s$ 
		\item $X_L = -N^2X_s$
		\item $P_{av_{max}} = \frac{V_s^2}{4R_s}$
		\item All above use RMS values
	\end{itemize}
\section{Resistors}
	\begin{itemize}
		\item Absorb average power ($\frac{VI}{2}$)
		\item No VAR (Q=0) 
		\item $\phi = 0$
		\item $S = I_{rms}^2R = \frac{V_{rms}^2}{R}$
		\item $v(t) = V\cos(\omega t)$
		\item $i(t) = I\cos(\omega t)$
		\item $p(t) = \frac{VI}{2}(1 + \cos(2\omega t))$
	\end{itemize}
\section{Inductors}
	\begin{itemize}
		\item No average power
		\item Absorbs VAR ($Q>0$)
		\item $\phi = +90\degree$
		\item $S = I_{rms}^2j\omega L = j\frac{V_{rms}^2}{\omega L}$
		\item $v(t) = V\cos(\omega t)$
		\item $i(t) = I\cos(\omega t - 90\degree)$
		\item $p(t) = \frac{VI}{2}(\sin(2\omega t))$
	\end{itemize}
\section{Capacitors}
	\begin{itemize}
		\item No average power
		\item Generates VAR ($Q<0$)
		\item $\phi = -90\degree$
		\item $S = -\frac{I_{rms}^2}{\omega C}j = -V_{rms}^2\omega Cj$
		\item $v(t) = V\cos(\omega t)$
		\item $i(t) = I\cos(\omega t + 90\degree)$
		\item $p(t) = -\frac{VI}{2}(\sin(2\omega t))$
	\end{itemize}
\section{Power Factor}
	\begin{itemize}
		\item $pf = \frac{P_{av}}{VI} = \frac{P_{av}}{|S|} = \cos\theta$
		\item $pf = 0 \implies$ Purely inductive/capacitive, no avg. dissipated power 
		\item $pf = 1 \implies$ Purely resistive load, all power is dissipated power
		\item $0 < pf < 1 \implies$ combination of resistive and reactive
		\item $\theta < 0 \implies$ leading and capacitive 
		\item $\theta > 0 \implies$ lagging and inductive
		\item $Q_A = -Q_B$
		\item $C = \frac{-Q_B}{\omega\widetilde{V_L}^2}$
		\item For power factor corrections, load type must be opposite of the part you can observe 
		\item $\theta < 0 \implies$ needs inductor and $\theta > 0 \implies$ needs capacitor
		\item $X_B = -\frac{R^2 + X_A^2}{X_A}$
		\item If $X_A = \omega L \implies X_B = -\frac{1}{\omega C}$
		\item If $X_A = \frac{-1}{\omega C} \implies X_B = \omega L$
	\end{itemize}
\section{Wye}
	\begin{itemize}
		\item $V_1 = \tilde{V}\angle 0\degree, V_2 = \tilde{V}\angle -120\degree, V_3 = \tilde{V}\angle -240\degree$
		\item Voltages are at the same frequency and have the same magnitude (sum of all three equals 0)
		\item Capital letters are for load nodes
		\item $\widetilde{V}_{Line} = \widetilde{V}_{Phase}*\sqrt{3}\angle 30\degree$
		\item $V_p = \frac{V_{line}}{\sqrt{3}}$ 
		\item $P = 3V_pI_p\cos\theta = \sqrt{3}V_{line}I_{line}\cos\theta$
	\end{itemize}
\section{Delta}
	\begin{itemize}
		\item $I_1 = \tilde{I}\angle 0\degree, I_2 = \tilde{I}\angle -120\degree, I_3= \tilde{I}\angle -240\degree$ 
		\item $\widetilde{I}_{Line} = \widetilde{I}_{Phase}*\sqrt{3}\angle -30\degree$
		\item $I_{phase} = \frac{I_{line}}{\sqrt{3}}$
		\item $P = 3V_pI_p\cos\theta = \sqrt{3}V_{line}I_{line}\cos\theta$
	\end{itemize}
\section{Misc.}
	\begin{itemize}
		\item $I^*$ indicates complex conjugate
		\item $S = \frac{V^2}{Z^*}$
		\item $R_{eq} \frac{1}{N^2}R_L$
		\item If power is greater than 0 the load is absorbing power
		\item A negative value needs an inductors while a positive value needs a capacitor
		\item Power factor correction reduces current
		\item Imaginary power describes storage and return of energy in inductors and capacitors
		\item If secondary is $\frac{1}{4}$ primary voltage, the secondary current is $4\times$ primary current
		\item pf = 1 for real power = apparent power
		\item LVDT measures position by changing magnetic coupling between primary and secondary coils
	\end{itemize}
\end{multicols*}
\newpage
ECE 2070 Equation Sheet \hfill Test 5 \\
\rule{7.6in}{0.4pt}
\begin{multicols*}{2}
	\section{Frequency Response}
	\begin{itemize}
		\item Gain = amplitude of output divided by amplitude of input
		\item Inductor impedance increases with frequency
		\item Capacitor impedance decreases with frequency
		\item Resistor impedance stays the same
		\item Filters shape the frequency response to perform specific operations
		\item Four types - low pass (lets low through), band pass (lets middle through), high pass (lets high through), notch (blocks middle)
		\item Cutoff frequency is the frequency where power is reduced to half its max value and gain is 0.707 of max
		\item Gain $= \frac{|V_c|}{|V|} = \frac{\omega_0L}{R} = \frac{1}{\omega_0CR}$
		\item Convert to dB - $20log_{10}\frac{V_{out}}{V_{in}}$
		\item Resonant frequency - $\omega_0 = \frac{1}{\sqrt{LC}}$
	\end{itemize}
	\section{Op Amps}
	\begin{itemize}
		\item Negative sign is inverting input, positive sign is non-inverting input, other side is output
		\item Inverting amplifier - $v_{out} = -\frac{R_F}{R_S}v_s$
		\item Non-inverting amplifier - $v_{out} = \left(1 + \frac{R_F}{R_S}\right)v_s$
		\item Buffer - $v_{out} = v_s$
		\item Differential - $v_{out} = \frac{R_2}{R_1}(v_2 - v_1)$
		\item Summing amplifier - $v_{out} = \sum\limits_{k=1}^n -\frac{R_F}{R_{Si}}v_{Sk}$
	\end{itemize}
	\section{Diodes \& Semiconductors}
	\begin{itemize}
		\item p-type semiconductor is doped so that the majority of the current is due to holes
		\item n-type semiconductor is doped so that majority of the current is due to electrons
		\item Zenor diode is breakdown, photo diode is reverse bias, LED is forward bias
		\item For p-type, hole moves in same direction of current
		\item Ideal diode dissipates no power
	\end{itemize}
	\section{Digital/Binary Systems}
	\begin{itemize}
		\item Analog signal can take any value in a range while digital signal can only take a finite number of values in a range
		\item To convert to binary, divide by two, add a 1 if there's a remainder and a 0 if clean division
		\item In logic tables, 0 is false and 1 is true
	\end{itemize}
		\section{Logic Gates and Digital Computation}
	\begin{itemize}
		\item Types of gates- AND, OR, NOT, NAND, NOR, XOR
		\item $A + \overline{A} = 1$
		\item $A \cdot \overline{A} = 0$
		\item $A \cdot A = A + A = A$
		\item $A + 1 = 1$
		\item $\overline{A + B} = \overline{A}\cdot\overline{B}$
		\item $\overline{A\cdot B} = \overline{A} + \overline{B}$
	\end{itemize}
	\newpage
	\section{Blank Space}
\end{multicols*}
\end{document}