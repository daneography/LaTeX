\documentclass [12pt] {article}
\usepackage{amsmath}
\usepackage{graphicx}
\usepackage{fancyhdr}
\usepackage{fancyvrb}
\setlength{\topmargin}{-0.75in}
\setlength{\textheight}{9.0in}
\setlength{\oddsidemargin}{-.25in}
\setlength{\textwidth}{7in}

\pagestyle{fancy}
\fancyhf{}
\rhead{KC Heinold \\
	Tom Hustedt \\
	Alec Lane \\
	Matt Trombley}
\lhead{MATH 3650 \\
Dr. Bowers}

\begin{document}
\hfill \break
\begin{center}
	\underline{\textbf{\huge{Lab 1}}}
\end{center}
1. Create a vector of the even whole numbers between 29 and 73. \\\\
\indent \verb|vec = [30:2:72];| \\ \\
2. Create a column vector with 11 equally spaced elements, whose first element is 2 and
whose last is 32. \\\\
\indent \verb|ans2=[2:3:32]';| \\\\
3. Let \verb|x = [2 5 1 6]|. 
\\\indent (a) Add 16 to each element 
\begin{Verbatim}[xleftmargin=.5in]
x=[2 5 1 6];
a=x+16
a =
    18    21    17    22
\end{Verbatim} \\
\\\indent (b) Add 3 to just the odd-index elements 
\begin{Verbatim}[xleftmargin=.5in]
x=[2 5 1 6];
b=x;
for i=1:2:length(x)
    b(i)=x(i)+3;
end
b
b =
     5     5     4     6
\end{Verbatim} \\
\\\indent (c) Compute the square root of each element 
\begin{Verbatim}[xleftmargin=.5in]
x=[2 5 1 6];
c=sqrt(x)
c =
    1.4142    2.2361    1.0000    2.4495
\end{Verbatim} \\
\\\indent (d) Compute the square of each element
\begin{Verbatim}[xleftmargin=.5in]
x=[2 5 1 6];
d=power(x,2)
d =
     4    25     1    36
\end{Verbatim} \\\\
4. Given \verb|x = [3 1 5 7 9 2 6]|, explain what the following commands “mean” by summarizing
the net result of the command.
\\\indent(a) \verb|x(3)| \\
\indent \indent This tells Matlab to extract the third number from the array. The result is \verb|ans=5|.\\
\\\indent(b) \verb x(1:7) \\
\indent \indent This tells Matlab to extract the first through seventh columns of the array. The result is the same array.
 \\
\\\indent(c) \verb x(1:end) \\
\indent \indent The code tells Matlab to extract the first column through the last column of the array. The “end’ tells it to use the last column and will apply to an array with any number of columns. The result is the same array.
 \\
\\\indent(d) \verb x(1:end-1) \\
\indent \indent This code tells Matlab to extract the first though second to last column of the array. The result is \verb|ans=[3 1 5 7 9 2]|.
\\
\\\indent(e) \verb x(6:-2:1) \\
\indent \indent This first tells Matlab to extract the sixth column of the array, then the fourth column, then second column. The result is \verb|ans=[2 7 1]|.
\\
\\\indent(f) \verb|x([1 6 2 1 1])| \\
\indent \indent This tells Matlab to extract the first, sixth, second, first, and first column of the array \verb|x| and put the values into an array in that order. The result is \verb|ans=[3 2 1 3 3]|.
\\
\\\indent(g) \verb sum(x) \\
\indent \indent This code tells Matlab to sum all of the values in a single row. The result is \verb|ans=33|.\\\\
5. Given the arrays \verb|x = [1 4 8]|, \verb|y = [2 1 5]| and \verb|A = [3 1 6 ; 5 2 7]|, determine which
of the following statements will correctly execute and provide a result. If the command
will not correctly execute, state why it will not.
\\\indent(a) \verb|x + y| \\
\par This will correctly run because we are adding two vectors of the same length, which outputs a new 1x3 vector.
\indent \begin{Verbatim}[xleftmargin=.5in]
x = [1 4 8];
y = [2 1 5];
ans =
     3     5    13
\end{Verbatim}
\\\indent(b) \verb|x + A| \\
\par This will not correctly run because the dimensions of the matrices are not
consistent.
\indent \begin{Verbatim}[xleftmargin=.5in]
x = [1 4 8];
A = [3 1 6;5 2 7];
x+A
Error using  + 
Matrix dimensions must agree.
\end{Verbatim}
\\\indent(c) \verb|x' + y| \\
\par This will not correctly run because the dimensions of the matrices are not
consistent.
\indent \begin{Verbatim}[xleftmargin=.5in]
x = [1 4 8];
y = [2 1 5];
x'+y
Error using  + 
Matrix dimensions must agree.
\end{Verbatim}
\\\indent(d) \verb|A - [ x' y' ]| \\
\par This will not correctly run because the dimensions of the matrices are not
consistent.
\indent \begin{Verbatim}[xleftmargin=.5in]
x = [1 4 8];
y = [2 1 5];
A = [3 1 6;5 2 7];
A-[x' y']
Matrix dimensions must agree.
\end{Verbatim}
\\\indent(e) \verb|[ x ; y' ]| \\
\par This will not correctly run because the dimensions of the matrices are not
consistent and are being concatenated.
\indent \begin{Verbatim}[xleftmargin=.5in]
x = [1 4 8];
y = [2 1 5];
[ x; y']
Error using vertcat
Dimensions of matrices being concatenated
are not consistent.
\end{Verbatim}
\\\indent(f) \verb|[ x ; y ]| \\
\par This will correctly run because the dimensions of the two vectors are the
 same, so they can be stacked ('concatenated') into a 2x3 matrix.
\indent \begin{Verbatim}[xleftmargin=.5in]
x = [1 4 8];
y = [2 1 5];
[ x; y]
ans =
     1     4     8
     2     1     5
\end{Verbatim}
\\\indent(g) \verb|A - 3|  \\
\par This will correctly run because the dimension of the matrix 'A' is not dependent on the number being subtracted, therefore 3 is subtracted from each value in A to produce a new 2x3 matrix.
\indent \begin{Verbatim}[xleftmargin=.5in]
A = [3 1 6;5 2 7];
A-3
ans =
     0    -2     3
     2    -1     4
\end{Verbatim}\\\\
6. Given \verb|x = [1 5 2 8 9 0 1]| and \verb|y = [5 2 2 6 0 0 2]|, execute and explain the results
of the following commands:
\\\indent(a) \verb|x > y|
\indent \begin{Verbatim}[xleftmargin=.5in]
x=[1 5 2 8 9 0 1]; 
y=[5 2 2 6 0 0 2];
ans6a= x > y;
ans6a =
     0     1     0     1     1     0     0
\end{Verbatim}
\\ \par The resulting matrix displays a 1 if the corresponding element in matrix \verb|x| is greater than the element in \verb|y| and a 0 if it is less than the value. \\  
\\\indent(b) \verb|y < x|
\indent \begin{Verbatim}[xleftmargin=.5in]
x=[1 5 2 8 9 0 1]; 
y=[5 2 2 6 0 0 2];
ans6b= y < x;
ans6b =
     0     1     0     1     1     0     0
\end{Verbatim}
\\\par This matrix displays the same results as the one in part (a). \\
\\\indent(c) \verb|x == y|
\indent \begin{Verbatim}[xleftmargin=.5in]
x=[1 5 2 8 9 0 1]; 
y=[5 2 2 6 0 0 2];
ans6c= x == y;
ans6c =
     0     0     1     0     0     1     0
\end{Verbatim}
\\\par The command puts a 1 in the matrix if corresponding elements in matrices \verb|x| and \verb|y| are equal, and a 0 if they are not. \\\newpage
\\\indent(d) \verb|x <= y|
\indent \begin{Verbatim}[xleftmargin=.5in]
x=[1 5 2 8 9 0 1]; 
y=[5 2 2 6 0 0 2];
ans6d= x <= y;
ans6d =
     1     0     1     0     0     1     1
\end{Verbatim}
\\\par The result of this command is a 1 in the matrix if the element in \verb|x| is less than or equal to \verb|y|, and a 0 if it is greater than the value in \verb|y|. \\
\\\indent(e) \verb|y >= x|
\indent \begin{Verbatim}[xleftmargin=.5in]
x=[1 5 2 8 9 0 1]; 
y=[5 2 2 6 0 0 2];
ans6e= y >= x;
ans6e =
     1     0     1     0     0     1     1
\end{Verbatim}
\\\par This matrix is very similar to the one in part (d). It produces the same result and the only difference is the ordering of the less than and equal symbols.\\
\\\indent(f) \verb|x |\mid\verb| y|$
\indent \begin{Verbatim}[xleftmargin=.5in]
x=[1 5 2 8 9 0 1]; 
y=[5 2 2 6 0 0 2];
ans6f= x | y;
ans6f =
     1     1     1     1     1     0     1
\end{Verbatim}
\\\par The code in this command is checking to see if there is a non-zero element in either matrix and returns a 1 if at least one corresponding element is not equal to zero. A value of 1 is returned if both values are 0. \\
\\\indent(g) \verb|x & y|
\indent \begin{Verbatim}[xleftmargin=.5in]
x=[1 5 2 8 9 0 1]; 
y=[5 2 2 6 0 0 2];
ans6g= x & y;
ans6g =
     1     1     1     1     0     0     1
\end{Verbatim} 
\\\par The command looks to see if there are elements in both \verb|x| and \verb|y|. If either matrix contains a 0 in an element the resulting matrix gets a 0, and if both have a number the matrix gets a 1. \\\newpage
\\\indent(h) \verb|x & (~y)|
\indent \begin{Verbatim}[xleftmargin=.5in]
x=[1 5 2 8 9 0 1]; 
y=[5 2 2 6 0 0 2];
ans6h= x & (~y);
ans6h =
     0     0     0     0     1     0     0
\end{Verbatim}
\\\par The code above checks to see if there is an element in matrix \verb|x| and not an element in matrix \verb|y|, which only appears in the fifth element. \\ 
\\\indent(i) \verb|(x > y)| $\mid$ \verb|(y < x)|
\indent \begin{Verbatim}[xleftmargin=.5in]
x=[1 5 2 8 9 0 1]; 
y=[5 2 2 6 0 0 2];
ans6i= (x > y) | (y < x);
ans6i =
     0     1     0     1     1     0     0
\end{Verbatim}
\\\par This command is essentially asking to find elements in \verb|x| that are greater than elements in \verb|y| twice. If the \verb|x| element is greater, a 1 is put in the resulting matrix; otherwise, a 0 is added. \\
\\\indent(j) \verb|(x > y) & (y < x)|
\indent \begin{Verbatim}[xleftmargin=.5in]
x=[1 5 2 8 9 0 1]; 
y=[5 2 2 6 0 0 2];
ans6j= (x > y) & (y < x);
ans6j =
     0     1     0     1     1     0     0
\end{Verbatim} \\ 
\par This command results in the same thing as part (i): a 1 if the element in \verb|x| is greater than the corresponding element in \verb|y|, and a 0 otherwise. \\\\
7. The exercises here show the techniques of logical-indexing (indexing with 0-1 vectors).
Given \verb|x = 1:10| and \verb|y = [3 1 5 6 8 2 9 4 7 0]|, execute and interpret the results of
the following commands:
\\\indent(a) \verb|(x > 3) & (x < 8)| 
\indent \begin{Verbatim}[xleftmargin=.5in]
x=1:10;
y=[3 1 5 6 8 2 9 4 7 0];
(x>3)&(x<8)
ans =
     0     0     0     1     1     1     1     0     0     0
\end{Verbatim} \\ 
\par This command creates a matrix of true/false values. (1 for true, 0 for false) It evaluates each term in the x-matrix to see if it is greater than 3 and less than 8. \\ \newpage
\\\indent(b) \verb|x(x > 5)| 
\indent \begin{Verbatim}[xleftmargin=.5in]
x=1:10;
y=[3 1 5 6 8 2 9 4 7 0];
x(x>5)
ans =
     6     7     8     9    10
\end{Verbatim} \\ 
\par This command creates a matrix of each indice value that is greater than five. \\
\\\indent(c) \verb|y(x <= 4)| 
\indent \begin{Verbatim}[xleftmargin=.5in]
x=1:10;
y=[3 1 5 6 8 2 9 4 7 0];
y(x<=4)
ans =
     3     1     5     6
\end{Verbatim} \\ 
\par This command finds the indices for each value in the x-matrix that are less than or equal to 4. It then creates a matrix of the y-values that correspond to each indice. \\
\\\indent(d) \verb|x( (x < 2)| $\mid$ \verb|(x >= 8) )| 
\indent \begin{Verbatim}[xleftmargin=.5in]
x=1:10;
y=[3 1 5 6 8 2 9 4 7 0];
x((x<2)|(x>=8))
ans =
     1     8     9    10
\end{Verbatim} \\ 
\par This command creates a matrix of each indice value in the x-matrix that is less than 2 or greater than or equal to 8. \\
\\\indent(e) \verb|y( (x >= 3) & (x < 6) )| 
\indent \begin{Verbatim}[xleftmargin=.5in]
x=1:10;
y=[3 1 5 6 8 2 9 4 7 0];
y((x>=3)&(x<6))
ans =
     5     6     8
\end{Verbatim} \\ 
\par This command finds the indices for each value in the x-matrix that are greater than or equal to 3 and less than 6. It then creates a matrix of the y-values that correspond to each one of these indices.\\\\ \newpage
8. Given \verb|x = [3 15 9 12 -1 0 -12 9 6 1]|, provide the command(s) that will
\\\indent(a) set the values of x that are negative to one \\
\indent \begin{Verbatim}[xleftmargin=.5in]
x = [3 15 9 12 -1 0 -12 9 6 1];
for i=1:length(x)
if x(i)<0
     x(i)=1
end
end
\end{Verbatim} \\ 
\\\indent(b) multiply the values of x that are even by 5 \\
\indent \begin{Verbatim}[xleftmargin=.5in]
x = [3 15 9 12 -1 0 -12 9 6 1];
for i=1:length(x)
if round(x(i)/2)==x(i)/2
     x(i)=x(i)*5
end
end
\end{Verbatim} \\ 
\\\indent(c) extract the values of x that are greater than 10 into a vector called y \\
\indent \begin{Verbatim}[xleftmargin=.5in]
x = [3 15 9 12 -1 0 -12 9 6 1];
p=0;
for i=1:length(x)
if x(i)>10
     p=p+1;
     y(p)=x(i);
end
end
\end{Verbatim} \\ 
\\\indent(d) set the values in x that are less than the mean to zero \\
\indent \begin{Verbatim}[xleftmargin=.5in]
x = [3 15 9 12 -1 0 -12 9 6 1];
for i=1:length(x)
if x(i)<mean(x)
     x(i)=0
end
end
\end{Verbatim} \\ 
\end{document}